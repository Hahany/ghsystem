% arara: pdflatex
% arara: biber
% arara: pdflatex
% arara: pdflatex
% --------------------------------------------------------------------------
% the GHSYSTEM package
%
%   globally harmonized system
%
% --------------------------------------------------------------------------
% Clemens Niederberger
% --------------------------------------------------------------------------
% https://github.com/cgnieder/ghsystem/
% contact@mychemistry.eu
% --------------------------------------------------------------------------
% If you have any ideas, questions, suggestions or bugs to report, please
% feel free to contact me.
% --------------------------------------------------------------------------
% Copyright 2011--2019 Clemens Niederberger
%
% This work may be distributed and/or modified under the
% conditions of the LaTeX Project Public License, either version 1.3
% of this license or (at your option) any later version.
% The latest version of this license is in
%   http://www.latex-project.org/lppl.txt
% and version 1.3 or later is part of all distributions of LaTeX
% version 2005/12/01 or later.
%
% This work has the LPPL maintenance status `maintained'.
%
% The Current Maintainer of this work is Clemens Niederberger.
% --------------------------------------------------------------------------
\documentclass[load-preamble+]{cnltx-doc}
\usepackage[utf8]{inputenc}
% \usepackage[greek=newtx]{chemmacros}
\usepackage{ghsystem}
\setcnltx{
  package  = {ghsystem},
  info     = \acl*{ghs} ,
  url      = https://github.com/cgnieder/ghsystem/ ,
  authors  = Clemens Niederberger ,
  email    = contact@mychemistry.eu ,
  abstract = {%
    \centering
    \includegraphics{chemmacros-logo.pdf}
    \par
  } ,
  add-cmds = {
    chemsetup,
    ghs, ghslistall, ghspic, ghssetup
  } ,
  index-setup = noclearpage
}

\usepackage{booktabs}

\expandafter\def\csname libertine@figurestyle\endcsname{LF}
\usepackage[libertine]{newtxmath}
\expandafter\def\csname libertine@figurestyle\endcsname{OsF}

\usepackage[biblatex]{embrac}
\ChangeEmph{[}[,.02em]{]}[.055em,-.08em]
\ChangeEmph{(}[-.01em,.04em]{)}[.04em,-.05em]

\usepackage[accsupp]{acro}
\acsetup{
  long-format  = \scshape ,
  short-format = \scshape
}
\DeclareAcronym{ghs}{
  short     = ghs ,
  long      = Globally Harmonized System of Classification and Labelling of
    Chemicals ,
  pdfstring = GHS ,
  accsupp   = GHS
}
\DeclareAcronym{eu}{
  short     = EU ,
  long      = European Union ,
  pdfstring = EU ,
  accsupp   = EU
}
\DeclareAcronym{UN}{
  short     = un ,
  long      = United Nations ,
  pdfstring = UN ,
  accsupp   = UN
}
\DeclareAcronym{dvi}{
  short     = dvi ,
  long      = device independent file format ,
  pdfstring = DVI ,
  accsupp   = DVO
}
\DeclareAcronym{pdf}{
  short     = pdf ,
  long      = portable document file ,
  pdfstring = PDF ,
  accsupp   = PDF
}

\chemsetup{
  greek = newtx ,
  formula = chemformula ,
  chemformula/format = \libertineLF
}

\ghssetup{
  language = {german,english}
}

\sisetup{
  detect-mode=false,
  mode=text,
  text-rm=\libertineLF
}

\usepackage{filecontents}

\defbibheading{bibliography}{\addsec{References}}
\addbibresource{\jobname.bib}
\begin{filecontents*}{\jobname.bib}
@misc{eu:ghsystem_regulation,
  author    = {{The European Parliament and The Council of the European Union}},
  title     = {Regulation (EC) No 1272/2008 of the European Parliament and of
    the Council} ,
  shorthand = {EuP} ,
  subtitle = {on classification, labelling and packaging of substances and
    mixtures, amending and repealing Directives 67/548/EEC and 1999/45/EC, and
    amending Regulation (EC) No 1907/2006} ,
  journal   = {Official Journal of the European Union} ,
  date      = {2008-12-16}
}
@online{unece:ghsystem_implementation,
  author   = {{United Nations Economic Commission for Europe}} ,
  title    = {GHS Implementation} ,
  url      =
    {http://www.unece.org/trans/danger/publi/ghs/implementation_e.html} ,
  urldate  = {2012-03-20} ,
  date     = {2012-03-20}
}
\end{filecontents*}

\newcommand*\tablehead[1]{\textrm{\bfseries#1}}

\begin{document}

\section{Introduction}
As a chemist you are probably aware of the fact that the \acl{UN} have
developed the \ac{ghs} as a global replacement for the various different
systems in different countries.  While it has not been implemented by all
countries yet~\cite{unece:ghsystem_implementation}, it is only a matter of
time.

The package \ghsystem{} now enables you to typeset all the hazard and
precautionary statements and pictograms in a very easy way.  The statements
are taken from \acs{eu} regulation 1272/2008~\cite{eu:ghsystem_regulation}.

\section{Licence and Requirements}
\license

\ghsystem\ loads the following packages:
\pkg{expl3}\footnote{\CTANurl{l3kernel}}~\cite{bnd:l3kernel}, \pkg{xparse} and
\pkg{l3keys2e}\footnote{\CTANurl{l3packages}}~\cite{bnd:l3packages},
\needpackage{chemmacros}~\cite{pkg:translations},
\needpackage{translations}~\cite{pkg:translations},
\needpackage{siunitx}~\cite{pkg:siunitx},
\needpackage{graphicx}~\cite{pkg:graphicx},
\needpackage{longtable}~\cite{pkg:longtable} and
\needpackage{ifpdf}~\cite{pkg:ifpdf}.

\section{Setup}
% TODO
% The simplest way is to load \pkg{chemmacros}~\cite{pkg:chemmacros} which loads
% \ghsystem{} implicitily.  All of \ghsystem's options belong to
% \pkg{chemmacros}' module \module{ghsystem}.  This means they can be setup with
% \begin{sourcecode}
%   \chemsetup[ghsystem]{<options>} or
%   \chemsetup{ghsystem/<option1>,ghsystem/<option2>}
% \end{sourcecode}
% \sinceversion{4.0}However, \ghsystem{} can be loaded as a standalone package
% and thus provides its own setup command:
% \begin{commands}
%   \command{ghssetup}[\marg{options}]
%     Setup command for \ghsystem.
% \end{commands}

\section{Get Hazard and Precautionary Statements}
\subsection{Simple Statements}
The general usage is simple: you use the command
\begin{commands}
  \command{ghs}[\sarg\oarg{options}\marg{type}\marg{number}]
    Get statement number \meta{number} of type \meta{type}.
\end{commands}
There are three types available: \code{h}, \code{euh} and \code{p}.  The
\meta{type} argument is case insensitive, so just type them in as you like.
\begin{example}[side-by-side]
  \ghs{h}{200} \par
  \ghs{H}{224} \par
  \ghs{euh}{001} \par
  \ghs{Euh}{202} \par
  \ghs{p}{201}
\end{example}

The starred version hides the identifier and only gives the statement.  If you
want to hide the statement itself instead you can use the option:
\begin{options}
  \keybool{hide}\Default{false}
    Hide the statement.
\end{options}

There is an option to customize the output, too.
\begin{options}
  \keyval{space}{space command}\Default
    Space between \meta{type} and \meta{number}.
\end{options}
\begin{example}[side-by-side]
  \ghs{h}{200} \par
  \ghs[space=\,]{h}{200} \par
  \ghs*{h}{200} \par
  \ghs[hide]{h}{200}
\end{example}

\subsection{Statements with Placeholders}
Some of the statements contain placeholders.  They can be one of the
following:
\begin{itemize}
  \item \textit{\textless state route of exposure if it is conclusively proven
      that no other routes of exposure cause the hazard\textgreater}
  \item \textit{\textless state specific effect if known\textgreater}
  \item \textit{\textless or state all organs affected, if known\textgreater}
  \item \textit{\textless name of sensitising substance\textgreater}
\end{itemize}

Except the last one which needs to be filled in, they are hidden per default.
They can be made visible with the option
\begin{options}
  \keybool{fill-in}\Default{false}
    Show placeholders.
\end{options}
\begin{example}
  \ghs{h}{340} \par
  \ghs[fill-in]{h}{340} \par
  \ghs{h}{360} \par
  \ghs[fill-in]{h}{360} \par
  \ghs{h}{370} \par
  \ghs[fill-in]{h}{370} \par
  \ghs{euh}{208} \par
  \ghs[fill-in]{euh}{208}
\end{example}

These placeholders can be replaced with one of these options:
\begin{options}
  \keyval{exposure}{text}\Default
    exposure placeholder
  \keyval{effect}{text}\Default
    effect placeholder
  \keyval{organs}{text}\Default
    organ placeholder
  \keyval{substance}{text}\Default
    substance placeholder
\end{options}
\begin{example}
  \ghs[exposure=This is how you get exposed.]{h}{340} \par
  \ghs[effect=These are the effects.]{h}{360} \par
  \ghs[organs=to this organ]{h}{370} \par
  \ghs[substance=substance]{euh}{208}
\end{example}

\subsection{Statements with Gaps}
Some of the statements have gaps that can be filled.
\begin{example}[side-by-side]
  \ghs{p}{301} \par
  \ghs{p}{401} \par
  \ghs{p}{411} \par
  \ghs{p}{413}
\end{example}

These gaps can be filled using these options:
\begin{options}
  \keyval{text}{text}
    Fill the \code{text} gap.
  \keyval{dots}{text}
    Fill the \code{dots} gap.
  \keyval{C-temperature}{num}
    Fill the Celsius temperature gap.
  \keyval{F-temperature}{num}
    Fill the Fahrenheit temperature gap.
  \keyval{kg-mass}{num}
    Fill the \si{\GHSkilogram} mass gap.
  \keyval{lbs-mass}{num}
    Fill the \si{\GHSpounds} mass gap.
\end{options}
\begin{example}
  \ghs[text=contact physician!]{p}{301} \par
  \ghs[dots=here]{p}{401} \par
  \ghs[C-temperature=50, F-temperature=122]{p}{411} \par
  \ghs[kg-mass=5.0, lbs-mass=11, C-temperature=50, F-temperature=122]{p}{413}
\end{example}

\subsection{Combined Statements}
There are some combinations of statements.  They are input with a \code{+}
between the numbers:
\begin{example}
  \ghs{p}{235+410} \\
  \ghs{p}{301+330+331}
\end{example}

Note that you can only get combinations that officially exist.  \emph{You
  can't combine freely}.

\section{Pictograms}
\subsection{The Pictures}
The \ac{ghs} defines a number of pictograms:

\ghspic{explos} \ghspic{flame} \ghspic{flame-O} \ghspic{bottle} \ghspic{acid}
\ghspic{skull} \ghspic{exclam} \ghspic{health} \ghspic{aqpol}

\begin{commands}
  \command{ghspic}[\oarg{options}\marg{name}]
    Load pictogram \meta{name}.
\end{commands}
Table~\ref{tab:ghs_pictograms} shows all available pictograms and their names.
To be more precise: it shows the names to use with the \cs{ghspic} command.
The file names are \code{ghsystem\_\meta{name}.\meta{filetype}} where
\meta{filetype} is \code{eps}, \code{pdf}, \code{jpg} or \code{png}, see also
section~\ref{ssec:picture_type}.
\begin{example}[side-by-side]
  \ghspic{skull}
\end{example}

If you don't like the default size you can change it using this option:
\begin{options}
  \keyval{scale}{factor}\Default{1}
    Scales the pictogram.
\end{options}
The pictures are actually quite large.  The default setting scales them by a
factor of $\frac{1}{20}$.
\begin{example}[side-by-side]
  \ghspic[scale=2]{skull}
\end{example}

If you want to use some specific \cs*{includegraphics} options, \eg, if
you want to rotate the pictogram for some reason, use this option:
\begin{options}
  \keyval{includegraphics}{includegraphics keyvals}
    Pass options to the underlying \cs*{includegraphics} command.
\end{options}
\begin{example}
  \ghspic[includegraphics={angle=90}]{skull}
\end{example}

\begin{longtable}{>{\ttfamily}ll>{\ttfamily}ll}
    \caption{All available \ac{ghs} pictograms.\label{tab:ghs_pictograms}} \\
  \toprule
    \normalfont\bfseries name & \bfseries pictogram &
    \normalfont\bfseries name & \bfseries pictogram \\
  \midrule\endfirsthead
  \toprule
    \normalfont\bfseries name & \bfseries pictogram &
    \normalfont\bfseries name & \bfseries pictogram \\
  \midrule\endhead
  \bottomrule\endfoot
  explos          & \ghspic{explos}          & explos-1        & \ghspic{explos-1} \\
  explos-2        & \ghspic{explos-2}        & explos-3        & \ghspic{explos-3} \\
  explos-4        & \ghspic{explos-4}        & explos-5        & \ghspic{explos-5} \\
  explos-6        & \ghspic{explos-6}        & & \\
  flame           & \ghspic{flame}           & flame-2-white   & \ghspic{flame-2-white} \\
  flame-2-black   & \ghspic{flame-2-black}   & flame-3-white   & \ghspic{flame-3-white} \\
  flame-3-black   & \ghspic{flame-3-black}   & flame-4-1       & \ghspic{flame-4-1} \\
  flame-4-2       & \ghspic{flame-4-2}       &
    flame-4-3-white & \ghspic{flame-4-3-white} \\
  flame-4-3-black & \ghspic{flame-4-3-black} &
    flame-5-2-white & \ghspic{flame-5-2-white} \\
  flame-5-2-black & \ghspic{flame-5-2-black} & & \\
  flame-O         & \ghspic{flame-O}         & flame-O-5-1     & \ghspic{flame-O-5-1} \\
  bottle          & \ghspic{bottle}          & bottle-2-black  & \ghspic{bottle-2-white} \\
  bottle-2-white  & \ghspic{bottle-2-black}  & & \\
  acid            & \ghspic{acid}            & acid-8          & \ghspic{acid-8} \\
  skull           & \ghspic{skull}           & skull-2         & \ghspic{skull-2} \\
  skull-6         & \ghspic{skull-6}         & & \\
  exclam          & \ghspic{exclam}          & & \\
  health          & \ghspic{health}          & & \\
  aqpol           & \ghspic{aqpol}           & & \\
\end{longtable}

\subsection{Picture Type Depending on Engine}\label{ssec:picture_type}
As you probably know you can't use every picture type with every compiler
engine.  \pdfTeX{} in \acs{dvi} mode \emph{needs} \code{eps} pictures while
\pdfTeX{} in \acs{pdf} mode, \XeTeX{} and \LuaTeX{} convert \code{eps}
pictures into \code{pdf} files, given they have the rights to write in the
directory the pictures are saved in.

However, the latter can include \code{jpg} and \code{png} without any
problems, while \pdfTeX{} in \acs{dvi} mode can't.

To resolve this \ghsystem\ tests which engine is used and if \pdfTeX{} which
mode is used and then chooses either \code{eps} or \code{pdf} for the
pictograms.  You are free to choose the picture type yourself with the option
\begin{options}
  \keychoice{pic-type}{eps,pdf,jpg,png}
    Choose the picture type.
\end{options}

\section{Available Languages}\label{sec:ghsystem_language}
Right now the H and P statements are available in English,
French\footnote{Thanks to Bréal Frédéric and Beaude Aurélien!}, German,
Italian\footnote{Thanks to Jonas Rivetti!} and Spanish\footnote{Thanks to
  Ignacio Fernández Galván!}.  The package adapts \pkg{chemmacros}' option
\option{language} or if the option hasn't been used recognizes the language
settings made with \pkg{babel} or \pkg{polyglossia}.  To be more precise: the
language selected at begin document is recognized.  Later changes won't affect
\ghsystem.  If you want to use different languages you have to use \ghsystem's
language option then.

You can also choose the language explicitly.
\begin{options}
  \keyval{language}{lang}\Default{english}
    Selects a language and if called in the preamble also loads the necessary
    language file if it hasn't been loaded, yet.  If the chosen file doesn't
    exist it falls back to  \code{english}.  Currently available choices are
    English, French, German, Italian, and Spanish.  \meta{lang} can be a comma
    separated list.  Then the last language in the list will be the active
    one.  If you plan to switch languages within the document then you should
    make sure to load all needed languages in the preamble first.
\end{options}
\begin{example}[side-by-side]
  \ghs{h}{201}

  \ghssetup{language=german}
  \ghs{h}{201}
\end{example}

% There is another alternative:
% \begin{commands}
%   \command{loadghsystemlanguage}[\marg{language}]
%     \sinceversion{4.0}Load the language used by \ghsystem.
% \end{commands}

I will add other languages some time in future.  This may take a while,
though.  If you would be willing to contribute and write the statements of
another language please feel free to contact
me\footnote{\href{mailto:contact@mychemistry.eu}{contact@mychemistry.eu}}.
Your \TeX\ distribution should contain a file
\code{ghsystem\_langtemplate.def} which \emph{should} explain all immediate
questions and can be used as a basis for a new language file.

\section{List of All Statements}
If for some reason you want to list all sentences you can use
\begin{commands}
  \command{ghslistall}[\oarg{options}]
    Print a table with all defined statements.
\end{commands}

This command has a number of options to customize the table, which is created
with the \env{longtable} environment of the \pkg{longtable} package.
\begin{options}
  \keyval{table-head-number}{text}\Default{Identifier}
    The table head for the number.
  \keyval{table-head-text}{text}\Default{Statement}
    The table head for the statement.
  \keyval{table-next-page}{text}\Default{continues on next page}
    The hint for a next page.
  \keyval{table-caption}{text}\Default{All H, EUH, and P Statements.}
    The \meta{text} in \cs*{caption}\marg{text}.
  \keyval{table-caption-short}{short text}\Default
    The \meta{short text} in \cs*{caption}\oarg{short text}\marg{text}.
  \keyval{table-label}{text}\Default{tab:ghs-hp-statements}
    The label to refer to the table with \cs*{ref} and similar commands.
  \keyval{table-row-sep}{dim}\Default{3pt}
    The separation of the table rows. A \TeX\ dimension.
  \keychoice{table-rules}{\default{default},booktabs,none}\Default{default}
    The style of the horizontal rules in the table.  \code{default} uses
    \cs*{hline}, \code{booktabs} uses \cs*{toprule}, \cs*{midrule} and
    \cs*{bottomrule}, resp.  This option needs the \pkg{booktabs} package
    which you have to load yourself then.
  \keychoice{table-top-head-rule}{\default{default},booktabs,none}\Default{default}
    Change top rule explicitly.
  \keychoice{table-head-rule}{\default{default},booktabs,none}\Default{default}
    Change rule below head explicitly.
  \keychoice{table-foot-rule}{\default{default},booktabs,none}\Default{default}
    Change foot rule explicitly.
  \keychoice{table-last-foot-rule}{\default{default},booktabs,none}\Default{default}
    Change last foot rule explicitly.
\end{options}

The code below shows how table~\ref{tab:ghs-hp-statements} was created:
\begin{sourcecode}
  \ghslistall[fill-in,table-rules=booktabs]
\end{sourcecode}

\ghslistall[fill-in,table-rules=booktabs]

\end{document}
